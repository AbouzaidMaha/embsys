\documentclass[12pt,t]{beamer}

%------------------------------------------------------------------------------
% configuration
%------------------------------------------------------------------------------
\RequirePackage{etex}
\usepackage{../../themes/dbt}
\usepackage{catchfilebetweentags}

\setbeameroption{hide notes}

\graphicspath{{images/}}

% a few macros
\newcommand{\bi}{\begin{itemize}}
\newcommand{\ei}{\end{itemize}}
\newcommand{\ig}{\includegraphics}
\newcommand{\myhref}[1]{\href{#1}{\tt \scriptsize #1}}
\newcommand{\incnote}[1]{\note{\ExecuteMetaData[notes.tex]{#1}}}

%------------------------------------------------------------------------------
% title
%------------------------------------------------------------------------------
% slide
\title{Systèmes d'exploitation pour l'embarqué}
\subtitle{UV 5.2 - Exécution et Concurrence}

\author{\href{}{Paul Blottière}}
\institute{
    \href{http://www.ensta-bretagne.fr/}{ENSTA Bretagne} \\[2pt]
    \href{}{\tt \scriptsize 10 Novembre 2015}
}
\date{
    \href{https://github.com/pblottiere}{\tt \scriptsize https://github.com/pblottiere} \\[2pt]
    \href{blottiere.paul@gmail.com}{\tt \scriptsize blottiere.paul@gmail.com}
}

% info
\begin{document}

{
\setbeamertemplate{footline}{} % no page number here
\frame{
    \titlepage
    \incnote{title}
} }

%------------------------------------------------------------------------------
% plan
%------------------------------------------------------------------------------
\begin{frame}{Plan}
    \subt{}
    \vspace{20pt}

    \bi
        \itemsep12pt
        \item Définitions
        \item Les commandes utiles
        \item Identification (PID, ...)
        \item Création de processus
        \item Phase d'init
        \item Terminaison, signaux et fichier core
        \item Les threads
    \ei

    \note {
    }
\end{frame}

%------------------------------------------------------------------------------
% def1
%------------------------------------------------------------------------------
\begin{frame}{Définition (1)}
    \subt{Programme et Processus}
    \vspace{20pt}

    \incnote{def1}
\end{frame}

%------------------------------------------------------------------------------
% def2
%------------------------------------------------------------------------------
\begin{frame}{Définition (2)}
    \subt{Process Control Block}
    \vspace{20pt}

    \incnote{def2}
\end{frame}

%------------------------------------------------------------------------------
% def3
%------------------------------------------------------------------------------
\begin{frame}{Définition (3)}
    \subt{États}
    \vspace{20pt}

    \incnote{def3}
\end{frame}

%------------------------------------------------------------------------------
% cmd1
%------------------------------------------------------------------------------
\begin{frame}{Les commandes utiles (1)}
    \subt{}
\end{frame}

\end{document}
