\documentclass[12pt,t]{beamer}

%------------------------------------------------------------------------------
% configuration
%------------------------------------------------------------------------------
\RequirePackage{etex}
\usepackage{../../themes/dbt}
\usepackage{catchfilebetweentags}

\setbeameroption{hide notes}

\graphicspath{{images/}}

% a few macros
\newcommand{\bi}{\begin{itemize}}
\newcommand{\ei}{\end{itemize}}
\newcommand{\ig}{\includegraphics}
\newcommand{\myhref}[1]{\href{#1}{\tt \scriptsize #1}}
\newcommand{\incnote}[1]{\note{\ExecuteMetaData[notes.tex]{#1}}}

%------------------------------------------------------------------------------
% title
%------------------------------------------------------------------------------
% slide
\title{Systèmes d'exploitation pour l'embarqué}
\subtitle{UV 5.2 - Exécution et Concurrence}

\author{\href{}{Paul Blottière}}
\institute{
    \href{http://www.ensta-bretagne.fr/}{ENSTA Bretagne} \\[2pt]
    \href{}{\tt \scriptsize 10 Novembre 2015}
}
\date{
    \href{https://github.com/pblottiere}{\tt \scriptsize https://github.com/pblottiere} \\[2pt]
    \href{blottiere.paul@gmail.com}{\tt \scriptsize blottiere.paul@gmail.com}
}

% info
\begin{document}

{
\setbeamertemplate{footline}{} % no page number here
\frame{
    \titlepage
    \incnote{title}
} }

%------------------------------------------------------------------------------
% plan
%------------------------------------------------------------------------------
\begin{frame}{Plan}
    \subt{}
    \vspace{20pt}

    \begin{enumerate}
        \itemsep12pt
        \item Définitions
        \item Outils d'analyse
        \item Identification (PID, ...)
        \item Capacités
        \item Création de processus
        \item Phase d'init
        \item Terminaison, signaux et fichier core
        \item Les threads
    \end{enumerate}

    \note {
    }
\end{frame}

%------------------------------------------------------------------------------
% def1
%------------------------------------------------------------------------------
\begin{frame}{Définition (1)}
    \subt{Programme et Processus}
    \vspace{20pt}

    \incnote{def1}
\end{frame}

%------------------------------------------------------------------------------
% def2
%------------------------------------------------------------------------------
\begin{frame}{Définition (2)}
    \subt{Ordonnanceur}
    \vspace{20pt}

    \incnote{def2}
\end{frame}

%------------------------------------------------------------------------------
% def3
%------------------------------------------------------------------------------
\begin{frame}{Définition (3)}
    \subt{Process Control Block}
    \vspace{20pt}

    \incnote{def3}
\end{frame}

%------------------------------------------------------------------------------
% def4
%------------------------------------------------------------------------------
\begin{frame}{Définition (4)}
    \subt{États}
    \vspace{20pt}

    \incnote{def4}
\end{frame}

%------------------------------------------------------------------------------
% def5
%------------------------------------------------------------------------------
\begin{frame}{Définition (5)}
    \subt{Limites}
    \vspace{20pt}

    \incnote{def5}
\end{frame}

%------------------------------------------------------------------------------
% analys1
%------------------------------------------------------------------------------
\begin{frame}{Outils d'Analyse (1)}
    \subt{Processus en cours}

    \incnote{analys1}
\end{frame}

%------------------------------------------------------------------------------
% analys2
%------------------------------------------------------------------------------
\begin{frame}{Outils d'analyse (2)}
    \subt{Divers commandes}
    \incnote{analys2}
\end{frame}

%------------------------------------------------------------------------------
% analys3
%------------------------------------------------------------------------------
\begin{frame}{Outils d'analyse (3)}
    \subt{/proc}
    \incnote{analys3}
\end{frame}

%------------------------------------------------------------------------------
% analys4
%------------------------------------------------------------------------------
\begin{frame}{Outils d'analyse (4)}
    \subt{/proc}
    \incnote{analys4}
\end{frame}

%------------------------------------------------------------------------------
% analys5
%------------------------------------------------------------------------------
\begin{frame}{Outils d'analyse (5)}
    \subt{/proc}
    \incnote{analys5}
\end{frame}

%------------------------------------------------------------------------------
% ident1
%------------------------------------------------------------------------------
\begin{frame}{Identification (1)}
    \subt{PID et PPID}
    \incnote{ident1}
\end{frame}

%------------------------------------------------------------------------------
% ident2
%------------------------------------------------------------------------------
\begin{frame}{Identification (2)}
    \subt{UID}
    \incnote{ident2}
\end{frame}

%------------------------------------------------------------------------------
% ident3
%------------------------------------------------------------------------------
\begin{frame}{Identification (3)}
    \subt{Groupe d'utilisateur, groupe de processus et groupe de groupe}
    \incnote{ident3}
\end{frame}

%------------------------------------------------------------------------------
% capa1
%------------------------------------------------------------------------------
\begin{frame}{Capacités d'un processus}
    \subt{Privilèges}
    \incnote{capa1}
\end{frame}

%------------------------------------------------------------------------------
% crea1
%------------------------------------------------------------------------------
\begin{frame}{Création de processus (1)}
    \subt{fork}
    \incnote{crea1}
\end{frame}

%------------------------------------------------------------------------------
% crea2
%------------------------------------------------------------------------------
\begin{frame}{Création de processus (2)}
    \subt{vfork}
    \incnote{crea2}
\end{frame}

%------------------------------------------------------------------------------
% crea3
%------------------------------------------------------------------------------
\begin{frame}{Création de processus (3)}
    \subt{exec, system et popen}
    \incnote{crea3}
\end{frame}

%------------------------------------------------------------------------------
% init1
%------------------------------------------------------------------------------
\begin{frame}{Phase d'init (1)}
    \subt{Kernel Thread}
    \incnote{init1}
\end{frame}

%------------------------------------------------------------------------------
% init2
%------------------------------------------------------------------------------
\begin{frame}{Phase d'init (2)}
    \subt{init}
    \incnote{init2}
\end{frame}

%------------------------------------------------------------------------------
% init3
%------------------------------------------------------------------------------
\begin{frame}{Phase d'init (3)}
    \subt{init toujours en vie...}
    \incnote{init3}
\end{frame}

%------------------------------------------------------------------------------
% end1
%------------------------------------------------------------------------------
\begin{frame}{Terminaison d'un programme (1)}
    \subt{Généralité}
    \incnote{end1}
\end{frame}

%------------------------------------------------------------------------------
% end2
%------------------------------------------------------------------------------
\begin{frame}{Terminaison d'un programme (2)}
    \subt{Tout va bien, je vais bien...}
    \incnote{end2}
\end{frame}

%------------------------------------------------------------------------------
% end3
%------------------------------------------------------------------------------
\begin{frame}{Terminaison d'un programme (3)}
    \subt{Terminaison anormale}
    \incnote{end3}
\end{frame}

%------------------------------------------------------------------------------
% end4
%------------------------------------------------------------------------------
\begin{frame}{Terminaison d'un programme (4)}
    \subt{Signaux}
    \incnote{end4}
\end{frame}

%------------------------------------------------------------------------------
% end5
%------------------------------------------------------------------------------
\begin{frame}{Terminaison d'un programme (5)}
    \subt{Signaux}
    \incnote{end5}
\end{frame}

%------------------------------------------------------------------------------
% end6
%------------------------------------------------------------------------------
\begin{frame}{Terminaison d'un programme (6)}
    \subt{Signaux}
    \incnote{end6}
\end{frame}

%------------------------------------------------------------------------------
% end7
%------------------------------------------------------------------------------
\begin{frame}{Terminaison d'un programme (7)}
    \subt{Signaux}
    \incnote{end7}
\end{frame}

%------------------------------------------------------------------------------
% pthread1
%------------------------------------------------------------------------------
\begin{frame}{Les PThreads (1)}
    \subt{Présentation}
    \incnote{pthread1}
\end{frame}

%------------------------------------------------------------------------------
% pthread2
%------------------------------------------------------------------------------
\begin{frame}{Les PThreads (2)}
    \subt{NTPL}
    \incnote{pthread2}
\end{frame}

%------------------------------------------------------------------------------
% pthread3
%------------------------------------------------------------------------------
\begin{frame}{Les PThreads (3)}
    \subt{Création et identification}
    \incnote{pthread3}
\end{frame}

%------------------------------------------------------------------------------
% pthread4
%------------------------------------------------------------------------------
\begin{frame}{Les PThreads (4)}
    \subt{Application}
    \incnote{pthread4}
\end{frame}

%------------------------------------------------------------------------------
% pthread5
%------------------------------------------------------------------------------
\begin{frame}{Les PThreads (5)}
    \subt{Terminaison}
    \incnote{pthread5}
\end{frame}

%------------------------------------------------------------------------------
% pthread6
%------------------------------------------------------------------------------
\begin{frame}{Les PThreads (6)}
    \subt{join}
    \incnote{pthread6}
\end{frame}

%------------------------------------------------------------------------------
% pthread7
%------------------------------------------------------------------------------
\begin{frame}{Les PThreads (7)}
    \subt{Nombre maximum de threads}
    \incnote{pthread7}
\end{frame}

%------------------------------------------------------------------------------
% pthread8
%------------------------------------------------------------------------------
\begin{frame}{Les PThreads (8)}
    \subt{detach}
    \incnote{pthread8}
\end{frame}

%------------------------------------------------------------------------------
% pthread9
%------------------------------------------------------------------------------
\begin{frame}{Les PThreads (9)}
    \subt{Thread Safety et \_REENTRANT}
    \incnote{pthread9}
\end{frame}

%------------------------------------------------------------------------------
% pthread10
%------------------------------------------------------------------------------
\begin{frame}{Les pthreads (10)}
    \subt{Synchronisation}
    \incnote{pthread10}
\end{frame}

%------------------------------------------------------------------------------
% pthread11
%------------------------------------------------------------------------------
\begin{frame}{Les pthreads (11)}
    \subt{Mutex et RW/Lock}
    \incnote{pthread11}
\end{frame}

%------------------------------------------------------------------------------
% ref
%------------------------------------------------------------------------------
\begin{frame}{Références}
\end{frame}

\end{document}
